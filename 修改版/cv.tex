\documentclass[11pt,a4paper]{moderncv}

\usepackage{fontspec,xunicode,xltxtra}

\defaultfontfeatures{Scale=MatchLowercase}
\setmainfont[Numbers=OldStyle,Mapping=tex-text]{Times New Roman}
\setsansfont[Mapping=tex-text]{Arial}
\setmonofont{Courier New}

\usepackage[BoldFont,SlantFont,CJKchecksingle,CJKnumber,CJKtextspaces]{xeCJK}
\setCJKmainfont[BoldFont={Adobe Heiti Std},ItalicFont={Adobe Kaiti Std}]{Adobe Song Std}
\setCJKsansfont{Adobe Heiti Std}
\setCJKmonofont{Adobe Fangsong Std}
\punctstyle{hangmobanjiao}
\defaultfontfeatures{Mapping=tex-text}
\XeTeXlinebreaklocale "zh"
\XeTeXlinebreakskip = 0pt plus 1pt minus 0.1pt

\usepackage{xcolor}
\linespread{1.2}

\moderncvtheme[blue]{classic}

\usepackage[scale=0.95]{geometry}
\AtBeginDocument{\recomputelengths}

\setCJKfamilyfont{name}{Adobe Kaiti Std}
\newcommand\name{\CJKfamily{name}}

\firstname{\name{贾艳成}}
\familyname{      }
\title{男/汉族/27岁 }

\mobile{手机:13408466755}
\email{邮箱:jiayancheng1@sina.com}
\homepage{个人博客:http://danieljyc.github.io/}
\photo[95pt]{picture3.jpg}
\quote{\textit{岗位意向:数据挖掘、推荐系统、图像处理等}}
%----------------------------------------------------------------------------------
%            content
%----------------------------------------------------------------------------------
\begin{document}
\maketitle


\section{个人概述}
\cvline{学术}{撰写4篇Paper,1项专利。}
\cvline{项目}{两个项目负责人。}
\cvline{竞赛}{阿里巴巴大数据竞赛(共7276支队伍),曾经连续数十天排名前10。}

\section{最高学历}
\cventry{2012年 -- 2015年}{硕士学位}{电子科技大学}{成都}{成绩:前30\%}{控制工程(研究方向:模式识别)}
\cvline{}{主要课程:矩阵理论、最优化理论与应用、随机过程及应用等。}

\section{个人技能}
\cvline{主要技能}{$\bullet$ 学术能力:撰写论文4篇(3篇投稿中);专利1项}
\cvline{}{$\bullet$ 编程语言:C, Python, SQL, C++, Java}
\cvline{}{$\bullet$ 数据挖掘算法:用逻辑回归、随机森林、GBRT、SVD++等处理天猫大数据}
\cvline{}{$\bullet$ 数据挖掘工具:ODPS-SQL(类似Hive)、ODPS-XLab(类似sas)}
\cvline{}{$\bullet$ 熟悉图像处理、识别的基本方法}
\cvline{}{$\bullet$ 能够对基本问题进行建模}

\cvline{次要技能}{$\bullet$ 用JS、HTML、PHP语言与他人合作编写微博应用;git/svn;用Markdown写个人博客(Hexo);习惯用思维导图整理思路。}
\cvline{}{$\bullet$ 了解Linux C、Linux Shell、LaTex等}
\cvline{}{$\bullet$ 其他(嵌入式):PCB、嵌入式、ZigBee}

\smallskip
\cvline{英语水平}{$\bullet$  六级:445}
\cvline{}{$\bullet$ 参与大数据专家组织的书籍翻译《Hadoop专业解决方案》;参考 Youtube、Coursera 英文教学视频。 }

\section{项目经历}
\cvline{2014年 -- 2015年}{山火识别算法及系统研制(进行中)}
\cvline{}{$\bullet$ 简要描述:360度采集全景图像,运用算法进行图像处理及山火识别。}
\cvline{}{$\bullet$ 职责:项目负责人。总体方案设计,图像处理及山火识别算法。}

\cvline{2013年 -- 2014年}{滑坡预测模型及系统研制(已完成)}
\cvline{}{$\bullet$ 简要描述:通过多种传感器采集滑坡体的数据信息,预处理后,利用Verhulst模型进行滑坡预警。 }
\cvline{}{$\bullet$ 职责:项目负责人。系统件设计,预警模型Verhulst。撰写3篇论文,1篇专利。}

\cvline{其他项目}{广场红绿灯空气质量监测预警系统(本科——校级项目——技术负责人)}

\section{竞赛奖励}
\cvline{阿里巴巴大数据竞赛(7276队)}{$\bullet$  简要描述:使用分布式平台ODPS,访问海量的天猫数据(5.7亿条,1000多万用户),并利用MapReduce、SQL及各种平台集成的机器学习算法包调试模型、提交结果。}
\cvline{}{$\bullet$ 职责:简单算法的设计及实现;利用ODPS-SQL、ODPS XLab、Python、Java等工具及语言。}
\cvline{}{$\bullet$ 曾连续数十天排名前10。}
\cvline{}{$\bullet$ 撰写1篇Paper。}
\cvline{电子设计竞赛}{$\bullet$ 省级二等奖}
\cvline{奖学金}{$\bullet$ 本科:2次二等,1次一等;研究生:1次二等。}

\section{自我评价}
\cvline{学习}{本科学到了一些“技术实力”;研究生在增强实力的同时,学到了一些“能力”,认识到了“想法很重要”。通过项目和竞赛学习到了一种方法:理论->实践->理论(Paper)。}
\cvline{性格}{待人诚恳;喜欢认识新朋友,很快融入新环境。}


\begin{thebibliography}{99}
\bibitem{1}\textbf Cao Yongxing, Xue Zhihang, Zhang Changhua, Jia Yancheng.Design and application of online landslide monitoring system for transmission lines corridor based on the optical fiber sensing technology. \textit{ICMMR2014}.
\bibitem{2}\textbf{}曹永兴,薛志航,张昌华,贾艳成,鲁庆华,蒋晨. (2013). 一种监测方法. \textit{专利申请号为2013106612030}.
\end{thebibliography}
\cvline{投稿中:}{一篇关于数据挖掘(推荐系统方向);两篇关于Verhulst预测模型。}
\end{document}



