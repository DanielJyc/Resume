\documentclass[11pt,a4paper]{moderncv}

\usepackage{fontspec,xunicode,xltxtra}

\defaultfontfeatures{Scale=MatchLowercase}
\setmainfont[Numbers=OldStyle,Mapping=tex-text]{Times New Roman}
\setsansfont[Mapping=tex-text]{Arial}
\setmonofont{Courier New}

\usepackage[BoldFont,SlantFont,CJKchecksingle,CJKnumber,CJKtextspaces]{xeCJK}
\setCJKmainfont[BoldFont={Adobe Heiti Std},ItalicFont={Adobe Kaiti Std}]{Adobe Song Std}
\setCJKsansfont{Adobe Heiti Std}
\setCJKmonofont{Adobe Fangsong Std}
\punctstyle{hangmobanjiao}
\defaultfontfeatures{Mapping=tex-text}
\XeTeXlinebreaklocale "zh"
\XeTeXlinebreakskip = 0pt plus 1pt minus 0.1pt

\usepackage{xcolor}
\linespread{1.2}

\moderncvtheme[blue]{classic}

\usepackage[scale=0.95]{geometry}
\AtBeginDocument{\recomputelengths}

\setCJKfamilyfont{name}{Adobe Kaiti Std}
\newcommand\name{\CJKfamily{name}}

\firstname{\name{贾艳成}}
\familyname{}
\title{个人简历}

\mobile{手机:13408466755}
\email{邮箱:jiayancheng1@sina.com}
\homepage{个人博客:http://danieljyc.github.io/}
\photo[95pt]{picture3.jpg}
\quote{\textit{求职意向:数据挖掘、算法、开发等}}

%----------------------------------------------------------------------------------
%            content
%----------------------------------------------------------------------------------
\begin{document}
\maketitle

\section{基本信息}
\cvcomputer{姓\qquad 名:}{贾艳成}{性\qquad 别:}{男}
\cvcomputer{民\qquad 族:}{汉族}{出\qquad 生:}{1987年01月06日}

\section{最高学历}
\cventry{2012年 -- 2015年}{硕士学位}{电子科技大学}{成都}{成绩:前30\%}{控制工程}
\cvline{}{主要课程:矩阵理论、最优化理论与应用、随机过程及应用等。}

\section{科研经历}
\cvline{2013年 -- 2014年}{滑坡预测模型及系统研制(已完成)}
\cvline{}{$\bullet$ 简要描述:通过各种传感器采集山体滑坡的数据信息,对数据处理后,利用Verhulst模型进行滑坡预警。 }
\cvline{}{$\bullet$ 职责:项目负责人。实现滑坡泥石流监测系统件设计,并用Verhulst模型进行预警。撰写3篇论文,一篇专利。}


\cvline{2014年 -- 2015年}{山火识别算法及系统研制(进行中)}
\cvline{}{$\bullet$ 简要描述:360度采集全景图像,传回终端,进行图像处理及山火识别。}
\cvline{}{$\bullet$ 职责:项目负责人。总体方案设计,图像处理及山火识别算法,软件设计。}

\cvline{其他经历}{广场红绿灯空气质量监测预警系统(本科——校级项目——技术负责人)}

\section{个人技能}
\cvline{专业技能}{$\bullet$ 熟悉分布式数据库Hive:用过类似平台ODPS-SQL}
\cvline{}{$\bullet$ 熟悉几种数据挖掘算法:用过逻辑回归、随机森林、SVD++处理大数据}
\cvline{}{$\bullet$ 用过ODPS-XLab做大数据分析:类似sas}
\cvline{}{$\bullet$ 用过Python、分布式SQL、Java:在大数据处理中使用过}
\cvline{}{$\bullet$ 用过简单的JS、HTML、PHP等网页相关知识:与日本同学合作编写微博应用}
\cvline{}{$\bullet$ 了解C、Linux C、嵌入式C、Linux Shell、C++、LaTex等}
\cvline{}{$\bullet$ 用过Github;用Markdown写个人博客(Hexo);思维导图。}
\cvline{}{$\bullet$ 次要技能:有一定的硬件开发实力:用过PCB、嵌入式知识}
\cvline{}{$\bullet$ 撰写并发表EI论文1篇(还有3篇投稿中);专利1篇。}

\smallskip
\cvline{英语水平}{$\bullet$  六级:445}
\cvline{}{$\bullet$ 阅读翻译能力:阅读英文专业资料;参与网络组织的翻译书籍《Hadoop专业解决方案》;参考 Youtube、Coursera 英文教学视频。 }

\section{竞赛情况}
\cvline{大数据竞赛}{$\bullet$ 职责:简单算法的设计及实现;利用ODPS-SQL(类似Hive)、ODPS XLab(类似sas)、Python、Java等工具及语言。}
\cvline{}{$\bullet$ 曾连续数十天排名前10。}
\cvline{}{$\bullet$ 撰写1篇Paper。}
\cvline{电子设计竞赛}{$\bullet$ 省级二等奖}

\section{自我评价}
\cvline{学习}{本科学到了一些“技术实力”;研究生在增强实力的同时,学到了一些“能力”,认识到了“想法很重要”}
\cvline{性格}{喜欢认识新朋友,与人进行沟通交流;}



\begin{thebibliography}{99}
\bibitem{1}\textbf Cao Yongxing, Xue Zhihang, Zhang Changhua, Jia Yancheng.Design and application of online landslide monitoring system for transmission lines corridor based on the optical fiber sensing technology. \textit{ICMMR2014}.
\bibitem{2}\textbf{}曹永兴,薛志航,张昌华,贾艳成,鲁庆华,蒋晨. (2013). 一种监测方法. \textit{专利申请号为2013106612030}.
\end{thebibliography}

\end{document}

